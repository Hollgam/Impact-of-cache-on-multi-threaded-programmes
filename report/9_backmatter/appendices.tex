\begin{appendices}

\chapter{Source code of \textit{pagefaults\_fopen.c}}
\label{app:listingTestPagefault}

\begin{lstlisting}[
language=C
]
#include <stdio.h>
#include <stdlib.h>
#include <unistd.h>
#include <string.h>

unsigned long long get_page_fault(int choice);
char * file_to_string(char *f);

struct proc_stats {
	int pid;			// %d
	char comm[256];		// %s
	char state;			// %c
	int ppid;			// %d
	int pgrp;			// %d
	int session;		// %d
	int tty_nr;			// %d
	int tpgid;			// %d
	unsigned long flags;	// %lu
	unsigned long long minflt;	// %lu
	unsigned long long cminflt;	// %lu
	unsigned long long majflt;	// %lu
	unsigned long long cmajflt;	// %lu
	unsigned long utime;	// %lu
	unsigned long stime; 	// %lu
	long cutime;		// %ld
	long cstime;		// %ld
	long priority;		// %ld
	long nice;			// %ld
	long num_threads;		// %ld
	long itrealvalue;		// %ld
	unsigned long starttime;	// %lu
	unsigned long vsize;	// %lu
	long rss;			// %ld
	unsigned long rlim;		// %lu
	unsigned long startcode;	// %lu
	unsigned long endcode;	// %lu
	unsigned long startstack;	// %lu
	unsigned long kstkesp;	// %lu
	unsigned long kstkeip;	// %lu
	unsigned long signal;	// %lu
	unsigned long blocked;	// %lu
	unsigned long sigignore;	// %lu
	unsigned long sigcatch;	// %lu
	unsigned long wchan;	// %lu
	unsigned long nswap;	// %lu
	unsigned long cnswap;	// %lu
	int exit_signal;		// %d
	int processor;		// %d
	unsigned long rt_priority;	// %lu
	unsigned long policy;	// %lu
	unsigned long long delayacct_blkio_ticks;	// %llu
};

int main(int argc, const char ** argv) {
	unsigned long long pageFaultsB = get_page_fault(1);

	// Read file
	FILE *fp;

	// Open file.
	if ((fp = fopen("/proc/interrupts", "r")) == NULL) {
		return (-1);
	}

	//Close the file if still open.
	if (fp) {
		fclose(fp);
	}
	// Finished reading file

	unsigned long long pageFaultsA = get_page_fault(1);

	printf("1st time /proc/interrupts: Before: %llu After:
	    %llu\n", pageFaultsB, pageFaultsA);

	pageFaultsB = get_page_fault(1);

	// Open file.
	if ((fp = fopen("/proc/interrupts", "r")) == NULL) {
		return (-1);
	}

	//Close the file if still open.
	if (fp) {
		fclose(fp);
	}
	// Finished reading file

	pageFaultsA = get_page_fault(1);

	printf("2nd time /proc/interrupts: Before: %llu After:
	    %llu\n", pageFaultsB, pageFaultsA);

	pageFaultsB = get_page_fault(1);

	// Open file.
	if ((fp = fopen("/proc/iomem", "r")) == NULL) {
		return (-1);
	}

	//Close the file if still open.
	if (fp) {
		fclose(fp);
	}
	// Finished reading file

	pageFaultsA = get_page_fault(1);

	printf("1st time /proc/iomem: Before: %llu After:
	    %llu\n", pageFaultsB, pageFaultsA);

	pageFaultsB = get_page_fault(1);

	// Open file.
	if ((fp = fopen("/proc/iomem", "r")) == NULL) {
		return (-1);
	}

	//Close the file if still open.
	if (fp) {
		fclose(fp);
	}
	// Finished reading file

	pageFaultsA = get_page_fault(1);

	printf("2nd time /proc/iomem: Before: %llu After:
	    %llu\n", pageFaultsB, pageFaultsA);

	char * str1 = file_to_string("/proc/interrupts");
	char * str2;


	while (strcmp(str1, str2) != 0) {
		pageFaultsB = get_page_fault(1);
		str2 = file_to_string("/proc/interrupts");
		pageFaultsA = get_page_fault(1);
	}

	printf("/proc/interrupts changed: Before: %llu After:
	    %llu\n", pageFaultsB, pageFaultsA);
}

int read_stat(char * filename, int pid, struct proc_stats *s) {
#ifndef __APPLE__
	const char *format =
		"%d %s %c %d %d %d %d %d %lu %lu %lu %lu %lu %lu %lu
		   %ld %ld %ld %ld %ld %ld %lu %lu %ld %lu %lu %lu
		   %lu %lu %lu %lu %lu %lu %lu %lu %lu %lu %d %d %lu
		   %lu %llu";
	FILE *fp;

	fp = fopen(filename, "r");
	if (fp) {
		if (42
		== fscanf(fp, format, &s->pid, s->comm, &s->state,
	    &s->ppid, &s->pgrp, &s->session, &s->tty_nr,
	    &s->tpgid, &s->flags, &s->minflt, &s->cminflt,
	    &s->majflt, &s->cmajflt, &s->utime, &s->stime,
		&s->cutime, &s->cstime, &s->priority, &s->nice,
		&s->num_threads, &s->itrealvalue,
		&s->starttime, &s->vsize, &s->rss, &s->rlim,
		&s->startcode, &s->endcode, &s->startstack,
		&s->kstkesp, &s->kstkeip, &s->signal, &s->blocked,
		&s->sigignore, &s->sigcatch, &s->wchan, &s->nswap,
		&s->cnswap, &s->exit_signal, &s->processor,
		&s->rt_priority, &s->policy, 
		&s->delayacct_blkio_ticks)) {
			if (fp) {
				fclose(fp);
			}
			return 1;
		} else {
			if (fp) {
				fclose(fp);
			}
			return 0;
		}
	} else {
		return 0;
	}
#else
	return -1;
#endif
}

// 1 - minor, 2 - major
unsigned long long get_page_fault(int choice) {
#ifndef __APPLE__
	struct proc_stats statsData;
	int self = getpid(); // Process ID

	char buf[256];
	sprintf(buf, "/proc/%d/stat", self);

	// Read data from the stats file
	read_stat(buf, self, &statsData);

	if (choice == 1) {
		return statsData.minflt;
//		return statsData->cminflt);

	} else if (choice == 2) {
		return statsData.majflt;
//	    return statsData->cmajflt;
	}

#else
	return -1;
#endif
}

// For storing contents of the file.
static char result[8][8 * 1024]; 
// Counter of how many files have been stored in result.
static int cycle = 0; 

char * file_to_string(char *f) {
	FILE *fp;
	char temp[512];

	cycle++;
	if (cycle == 8)
		cycle = 0;

	// Open file.
	if ((fp = fopen(f, "r")) == NULL) {
		return (char *) (-1);
	}

	// Initialise the result
	result[cycle][0] = 0;

	// Search for str and extract numeric.
	while (fgets(temp, 512, fp) != NULL) {
		strcat(result[cycle], temp);
	}

	// Check if there is a memory leak. 
	if (strlen(result[cycle]) > 8 * 1024) {
		printf("Memory error - strlen(result)==%lu,
		    file size==%d\n", strlen(result[cycle]),
		    8 * 1024);
		if (fp) {
			fclose(fp);
		}
		exit(1);
	}

	//Close the file if still open.
	if (fp) {
		fclose(fp);
	}

	return &result[cycle][0];
}
\end{lstlisting}

\chapter{Source code of \textit{clock-gettime\_test.c}}
\label{app:listingTestClockGetTime}

\begin{lstlisting}[
language=C
]
#include <stdio.h>
#include <stdlib.h>
#include <unistd.h>
#include <signal.h>
#include <inttypes.h>
#include <cpuid.h>
#include <sys/prctl.h>
#include <linux/prctl.h>
#include <time.h>

// Get the process' ability to use the
// timestamp counter instruction
#ifndef PR_GET_TSC
#define PR_GET_TSC 25
#define PR_SET_TSC 26
// Allow the use of the timestamp counter
# define PR_TSC_ENABLE 1 
// Throw a SIGSEGV instead of reading the TSC
# define PR_TSC_SIGSEGV	2 
#endif

#define LOOPS 1024

const char *tsc_names[] = { [0] = "[not set]",
    [PR_TSC_ENABLE] = "PR_TSC_ENABLE",
    [PR_TSC_SIGSEGV] = "PR_TSC_SIGSEGV", };

void sigsegv_cb(int sig) {
	int tsc_val = 0;

	printf("[ SIG_SEGV ]\n");
	printf("prctl(PR_GET_TSC, &tsc_val); ");
	fflush(stdout);

	if (prctl(PR_GET_TSC, &tsc_val) == -1)
		perror("prctl");

	printf("tsc_val == %s\n", tsc_names[tsc_val]);
	printf("prctl(PR_SET_TSC, PR_TSC_ENABLE)\n");
	fflush(stdout);
	if (prctl(PR_SET_TSC, PR_TSC_ENABLE) == -1)
		perror("prctl");

	printf("clock_gettime() == ");
}

int main(int argc, char **argv) {
	int tsc_val = 0;
	struct timespec r1, r2, r3, r4, temp;
	struct timespec rs[LOOPS];
	int i, j;
    
    // Make sure there is no I/O pending from
    // this process
	sleep(1); 
	get_time(&r1);
	get_time(&r2);
	get_time(&r3);
	get_time(&r4);
	// This (might) ensure that we have a full
	// time quantum to execute in - as we get
    // re-scheduled after the sleep
	usleep(10); 
	// the next few instructions get pre-loaded
	// into i-cache
	get_time(&r1);
	get_time(&r2);
	get_time(&r3);
	get_time(&r4);

	printf("%llu %llu %llu %llu\n", (unsigned long long)
	    r1.tv_nsec, (unsigned long long) r2.tv_nsec, 
	    (unsigned long long) r3.tv_nsec,
	    (unsigned long long) r4.tv_nsec);
	printf("%llu %llu %llu\n", (unsigned long long)
	    (r2.tv_nsec - r1.tv_nsec), (unsigned long long)
	    (r3.tv_nsec - r2.tv_nsec), (unsigned long long)
	    (r4.tv_nsec - r3.tv_nsec));

	printf("clock_gettime() == ");
	fflush(stdout);

    // Make sure there is no I/O pending from this process
	sleep(1); 
	// Use 2 loops to preload the i-cache and makes sure 
	// there will be no page faults on the rs array
	for (j = 0; j < 2; j++) {
		for (i = 0; i < LOOPS; i++) {
			get_time(&rs[i]);
		}
	}
	for (i = 1; i < LOOPS; i++)
		printf("%llu ", (unsigned long long)
		    rs[i].tv_nsec - rs[i - 1].tv_nsec);

	printf("\n");
	fflush(stdout);
	exit(EXIT_SUCCESS);
}

//  Get time in nano-seconds
int get_time(struct timespec *timeStruct) {
	if (clock_gettime(CLOCK_MONOTONIC, timeStruct) == -1) {
		perror("clock getres");
		return 0;
	}
	return 1;
}
\end{lstlisting}

\chapter{Source code of the function \textit{void test\_rdtsc(void)}}
\label{app:listingTestRDTSC}

\begin{lstlisting}[
language=C
]
// Test rdtsc
void test_rdtsc(void) {
	// With CPUID
	printf("TEST OF RDTSC with CPUID\n");

	unsigned long long t[32], prev;
	int i;
	for (i = 0; i < 32; i++)
		t[i] = rdtsc_old(1);

	prev = t[0];
	for (i = 1; i < 32; i++) {
		printf("%llu [%llu]\n", t[i], t[i] - prev);
		prev = t[i];
	}

	printf("Total=%llu\n", t[32 - 1] - t[0]);

	// Without CPUID
	printf("\nTEST OF RDTSC without CPUID\n");

	for (i = 0; i < 32; i++)
		t[i] = rdtsc_old(0);

	prev = t[0];
	for (i = 1; i < 32; i++) {
		printf("%llu [%llu]\n", t[i], t[i] - prev);
		prev = t[i];
	}

	printf("Total=%llu\n", t[32 - 1] - t[0]);
}
\end{lstlisting}

\chapter{Bash code for running lmbench on the Xeon E5-2695 v2}
\label{app:listingLmbenchICHEC}

\begin{lstlisting}[
language=bash
]
#!/bin/bash
#PBS -l nodes=1:ppn=24
#PBS -l walltime=5:00:00
#PBS -N lmbench
#PBS -A nuim01
#PBS -r n
#PBS -j oe
#PBS -m bea
#PBS -M pavlo.bazilinskyy@gmail.com

cd $PBS_O_WORKDIR
OS=x86_64-linux-gnu
CONFIG=CONFIG.r1i1n5
RESULTS=results/$OS
BASE=../$RESULTS/`uname -n`
EXT=0

if [ ! -f "../bin/$OS/$CONFIG" ]
then    echo "No config file?"
        exit 1
fi
. ../bin/$OS/$CONFIG

if [ ! -d ../$RESULTS ]
then    mkdir -p ../$RESULTS
fi
RESULTS=$BASE.$EXT
while [ -f $RESULTS ]
do      EXT=`expr $EXT + 1`
        RESULTS=$BASE.$EXT
done

cd ../bin/$OS
PATH=.:${PATH}; export PATH
export SYNC_MAX
export OUTPUT
lmbench $CONFIG 2>../${RESULTS}

if [ X$MAIL = Xyes ]
then    echo Mailing results
        (echo ---- $RESULTS ---
        cat ../$RESULTS) | mail pavlo.bazilinskyy@gmail.com
fi
exit 0
\end{lstlisting}


\chapter{Source code of the main function of \textit{void test\_time\_int\_pf.c}}
\label{app:listingTestInterrupt}

\begin{lstlisting}[
language=C
]
int main(int argc, const char ** argv) {
	int k;
	long sum;
	struct timespec start, stop;
	int run = 10;

	// Record times of experiments in the run.
	unsigned long long *time = malloc(sizeof(
	    unsigned long long) * run);
	if (time == NULL) {
		printf("Error with allocating space for the array\n");
		exit(1);
	}

	// Caculate the average duration of an interrupt
	printf("INTERRUPTS\n\ntime,num\n");
	int i = 0;
	for (i = 0; i < run; ++i) {
		unsigned long long interruptsBefore;
		unsigned long long interruptsAfter;

		// Warmup
		interruptsAfter = search_in_file(
		    "/proc/interrupts", "LOC:", 1);
		interruptsBefore = interruptsAfter;

		while (interruptsBefore == interruptsAfter)
			interruptsBefore = search_in_file("
			    /proc/interrupts", "LOC:", 1);

		// Create interrupts
		do {
		    // Record time before causing the interrupt
			get_time_ns(&start);
			interruptsAfter = search_in_file(
			    "/proc/interrupts", "LOC:", 1);
		} while (interruptsAfter - interruptsBefore != 1);
        
        // Record time after causing the interrupt
		get_time_ns(&stop);
    
        // How many interrupts occurred
		int numInterrupts = interruptsAfter -
		    interruptsBefore;
		    
		// Record time with interrupts
		unsigned long long timeWithInterupts =
		    calculate_time_ns(start, stop);

		printf("%d,%llu\n", timeWithInterupts, numInterrupts);
		// Record time difference over a number of interrupts
		time[i] = timeWithInterupts / numInterrupts;
	}
	printf("1 interrupt takes (average from %d runs):
	    %llu\n", run, average_time(time, run));

	// Caculate the average duration of a page fault
	printf("\nPAGE FAULTS\n\ntime,num\n");
	for (i = 0; i < run; ++i) {
		unsigned long long pfBefore;
		unsigned long long pfAfter;

		// Warmup
		struct proc_stats stat_file =
		    get_page_fault_file();
		pfAfter = get_page_fault(stat_file, 1);
		pfBefore = pfAfter;

		while (pfBefore == pfAfter) {
			stat_file = get_page_fault_file();
			pfBefore = get_page_fault(stat_file, 1);
		}
        
        // Record time before causing a page fault
		get_time_ns(&start);

		// Create page faults
		do {
			stat_file = get_page_fault_file();
			pfAfter= get_page_fault(stat_file, 1);
		}while (pfAfter - pfBefore == 0);
    
        // Record time after causing a page fault
		get_time_ns(&stop); 

        // How many minor page faults occurred
		int numPf = pfAfter - pfBefore;
		// Record time with page faults
		unsigned long long timeWithPf =
		    calculate_time_ns(start, stop);
    
        // Record time difference over a number of page faults
		printf("%d,%llu\n", timeWithPf, numPf);
		time[i] = timeWithPf / numPf;
	}
	printf("1 interrupt takes (average from %d runs):
	    %llu\n", run, average_time(time, run));

	free(time);
	return sum;
}
}

\end{lstlisting}


\chapter{Source code of the experiment with threads residing on the same core}
\label{app:listingExperiment2}

\begin{lstlisting}[
language=C
]
/*
 * EXPERIMENT 2
 *
 */
void experiment_2(int n) {
	// Aligned array for manipulating data
	long *testAr = align_long_array(sizeof(long) * n);

	// Wrap information that has to be passed to a pthread
	struct argStructType * argStruct = malloc(sizeof(
	    struct argStructType));
	argStruct->experimentId = 2;
	argStruct->n = n;
	argStruct->testAr = testAr;

	pthread_mutex_init(&mut, NULL); // Initialise the mutex.

	// Create pthreads
	rc = pthread_create(&thread1, NULL, e2_pthread_main1,
	    (void *) argStruct);
	if (rc) {
		printf("ERROR; return code from pthread_create()
		    is %d\n", rc);
		exit(-1);
	}

	// Join threads
	pthread_join(thread1, NULL);
	pthread_join(thread2, NULL);

	// Finish
	free(testAr);
	// Free memory allocated for generic argument struct
	free(argStruct); 
	pthread_mutex_destroy(&mut);
}

// Sender. This thread sends data
void *e2_pthread_main1(void * argStruct) {
    // Pin to the first core of the first CPU.
	pin_thread_to_core(0); 
    
    // Unpack arguments
	struct argStructType *args =
	    (struct argStructType *) argStruct; 	

    // Lock mutex.
	pthread_mutex_lock(&mut); 

    // Create the 2nd thread
	rc = pthread_create(&thread2, NULL, e2_pthread_main2,
	    (void *) argStruct);
	if (rc) {
		printf("ERROR; return code from pthread_create()
		    is %d\n", rc);
		exit(-1);
	}

	// Work with  shared data
	int i = 0;
	for (i = 0; i < args->n; ++i) {
		args->testAr[i] = 3l;
	}

	pthread_mutex_unlock(&mut); // Lock mutex.

	return ((void *) 1);
}

// Receiver. This thread receives data
void *e2_pthread_main2(void * argStruct) {
    // Pin to the first core of the first CPU.
	pin_thread_to_core(0); 
    
    // Unpack arguments
	struct argStructType *args =
	    (struct argStructType *) argStruct;

    // Lock mutex.
	pthread_mutex_lock(&mut); 

	// Work with  shared data
	int i = 0;
	long temp; // For assignment of values.
	for (i = 0; i < args->n; ++i) {
		temp = args->testAr[i];
	}
    
    // Lock mutex.
	pthread_mutex_unlock(&mut); 

	return ((void *) 1);
}

\end{lstlisting}

\chapter{Makefile used to run experiments}
\label{app:listingMakefile}

\begin{lstlisting}[
language=bash
]
CC =gcc
IFLAGS =-I
WFLAG1 = -Wall
WFLAG2 = -Werror
WFLAG3 = -Wextra
WFLAGS = $(WFLAG1)
OFLAGS = -g -O0
DFLAGS = # -Doptions
UFLAGS = # Set on make command line only
CFLAGS = $(SFLAGS) $(DFLAGS) $(IFLAGS) $(OFLAGS) $(WFLAGS) $(UFLAGS)
LIBS =
DEPS =test.h hr_timer.h file_worker.h conf.h experiments.h test_env.h
OBJ  =test.o hr_timer.o file_worker.o experiments.o test_env.o

UNAME_S := $(shell uname -s)
ifeq ($(UNAME_S),Linux)
	LIBS += -lrt -lpthread -lm
endif
ifeq ($(UNAME_S),Darwin)
	DEPS += clock_gettime_mac.h
	OBJ  += clock_gettime_mac.o
endif

%.o: %.c $(DEPS)
	$(CC) -c -g -o $@ $< $(CFLAGS) $(LIBS)

test: $(OBJ)
	$(CC) -o $@ $^ $(LIBS)

.PHONY: clean

clean:
	rm -f *.o *~
\end{lstlisting}

\chapter{Average duration of interrupts and minor page faults, Xeon 5130}
\label{app:durationIntNuim}

\begin{lstlisting}[
language=bash
]
INTERRUPTS
time,num
84547,1
154798,1
84397,1
84743,1
84797,1
96676,1
101176,1
650527,1
95336,1
99946,1
1 interrupt takes (average from 10 runs): 153694

PAGE FAULTS
time,num
42207,1
41603,1
41320,1
47684,1
41624,1
41336,1
41315,1
46944,1
41336,1
41293,1
1 page fault takes (average from 10 runs): 42666
\end{lstlisting}

\chapter{Average duration of interrupts and minor page faults, Xeon E5-2695 v2}
\label{app:durationIntIchec}

\begin{lstlisting}[
language=bash
]
INTERRUPTS
time,num
315246,1
332646,1
313228,1
314809,1
319510,1
314532,1
315445,1
318588,1
368649,1
315614,1
1 interrupt takes (average from 10 runs): 322826
PAGE FAULTS
time,num
19733,1
26440,1
19987,1
23632,1
18098,1
23189,1
18420,1
18427,1
25780,1
23624,1
1 page fault takes (average from 10 runs): 21733
\end{lstlisting}

\chapter{Results of Experiment 1, filtered data, Xeon 5130}
\label{app:cycle-level-results-clean-nuim}

Information on the numbers of interrupts and minor page faults for first three runs of each iteration of the experiment is given. Selected rows were removed. The original file that contains information on all 10 runs of each of 235 sub-experiments of each iteration of the experiment and presents data on the numbers of recorded major page faults, may be found at: \url{https://github.com/Hollgam/cache-mt/tree/master/results/nuim\_clean-1-0.csv}.

\csvautolongtable{9_backmatter/xeon_clean-1-0.csv}

\chapter{Results of Experiment 1, filtered data, Xeon E5-2695 v2}
\label{app:cycle-level-results-clean-ichec}

Information on the numbers of interrupts and minor page faults for first three runs of each iteration of the experiment is given. Selected rows were removed. The original file that contains information on all 10 runs of each of 235 sub-experiments of each iteration of the experiment and presents data on the numbers of recorded major page faults, may be found at: \url{https://github.com/Hollgam/cache-mt/tree/master/results/ichec\_clean-1-0.csv}.

\csvautolongtable{9_backmatter/ichech_xeon_clean-1-0.csv}

\chapter{Results of running the memory benchmark from lmbench}
\label{app:lmbench-results}

\csvautolongtable{9_backmatter/lmbench.csv}

\chapter{Results of Experiment 2, unfiltered data, Xeon 5130}
\label{app:app-level-results-2-nuim}

Information on the numbers of interrupts and minor page faults for first three runs of each iteration of the experiment is given. Selected rows were removed. The original file that contains information on all 10 runs of each of 235 sub-experiments of each iteration of the experiment and presents data on the numbers of recorded major page faults, may be found at: \url{https://github.com/Hollgam/cache-mt/tree/master/results/nuim\_dirty-2-0.csv}.

\csvautolongtable{9_backmatter/xeon_dirty-2-0.csv}

\chapter{Results of Experiment 2, unfiltered data, Xeon E5-2695 v2}
\label{app:app-level-results-2-ichec}

Information on the numbers of interrupts and minor page faults for first three runs of each iteration of the experiment is given. Selected rows were removed. The original file that contains information on all 10 runs of each of 235 sub-experiments of each iteration of the experiment and presents data on the numbers of recorded major page faults, may be found at: \url{https://github.com/Hollgam/cache-mt/tree/master/results/ichec\_dirty-2-0.csv}.

\csvautolongtable{9_backmatter/ichech_xeon_dirty-2-0.csv}

\chapter{Results of Experiment 3, unfiltered data, Xeon 5130}
\label{app:app-level-results-3-nuim}

Information on the numbers of interrupts and minor page faults for first three runs of each iteration of the experiment is given. Selected rows were removed. The original file that contains information on all 10 runs of each of 235 sub-experiments of each iteration of the experiment and presents data on the numbers of recorded major page faults, may be found at: \url{https://github.com/Hollgam/cache-mt/tree/master/results/nuim\_dirty-3-0.csv}.

\csvautolongtable{9_backmatter/xeon_dirty-3-0.csv}

\chapter{Results of Experiment 3, unfiltered data, Xeon E5-2695 v2}
\label{app:app-level-results-3-ichec}

Information on the numbers of interrupts and minor page faults for first three runs of each iteration of the experiment is given. Selected rows were removed. The original file that contains information on all 10 runs of each of 235 sub-experiments of each iteration of the experiment and presents data on the numbers of recorded major page faults, may be found at: \url{https://github.com/Hollgam/cache-mt/tree/master/results/ichec\_dirty-3-0.csv}.

\csvautolongtable{9_backmatter/ichech_xeon_dirty-3-0.csv}

\chapter{Results of Experiment 4, unfiltered data, Xeon 5130}
\label{app:app-level-results-4-nuim}

Information on the numbers of interrupts and minor page faults for first three runs of each iteration of the experiment is given. Selected rows were removed. The original file that contains information on all 10 runs of each of 235 sub-experiments of each iteration of the experiment and presents data on the numbers of recorded major page faults, may be found at: \url{https://github.com/Hollgam/cache-mt/tree/master/results/nuim\_dirty-4-0.csv}.

\csvautolongtable{9_backmatter/xeon_dirty-4-0.csv}

\chapter{Results of Experiment 4, unfiltered data, Xeon E5-2695 v2}
\label{app:app-level-results-4-ichec}

Information on the numbers of interrupts and minor page faults for first three runs of each iteration of the experiment is given. Selected rows were removed. The original file that contains information on all 10 runs of each of 235 sub-experiments of each iteration of the experiment and presents data on the numbers of recorded major page faults, may be found at: \url{https://github.com/Hollgam/cache-mt/tree/master/results/ichec\_dirty-4-0.csv}.

\csvautolongtable{9_backmatter/ichech_xeon_dirty-4-0.csv}

\end{appendices}